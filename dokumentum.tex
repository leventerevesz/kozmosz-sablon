%--------------------------------------------------------------------------------------
% Dokumentum formátuma
%--------------------------------------------------------------------------------------
\documentclass[12pt]{article}

%--------------------------------------------------------------------------------------
% Változók beállítása
%--------------------------------------------------------------------------------------

%TODO Állítsd be az alábbi változókat

% Szerző
\def\szerzoA {Gyenge Ákos}
\def\szerzoB {Takács Donát}
\def\szerzoC {}
\def\szerzoD {}

% Munka címe
\def\cim {Projektfeladat beszámoló}
\def\alcim {Cubesat tervezése}

\def\hely {Budapest}
\def\datum {\today}


%--------------------------------------------------------------------------------------
% Inicializáció (valószínűleg nem kell hozzányúlni)
%--------------------------------------------------------------------------------------
% Csomagok betöltése

\usepackage[english,magyar]{babel}      % Alapértelmezés szerint utoljára definiált 
% nyelv lesz aktív, de később külön beállítjuk 
% az aktív nyelvet.


%\usepackage{cmap}
\usepackage{amsfonts,amsmath,amssymb}   % Matematikai szimbólumok
\usepackage{graphicx}

\usepackage{geometry}                   % Margók
\usepackage{setspace}                   % Sorköz
\usepackage{booktabs}                   % Jó táblázatok
\usepackage{fancyhdr}                   % Jó élőfejek

\usepackage[unicode]{hyperref}          % Linkek
\usepackage{xcolor}                     % Syntax highlihting
\usepackage{listings}                   % Forráskódokhoz
\usepackage[hang]{caption}              % Jobb képfeliratok
\usepackage[numbers]{natbib}            % Bibliográfia stílus
\usepackage{xspace}                     % Helykihagyás
\usepackage{ifthen}


% Karakterkódolás beállítása (xetex - lualatex és pdflatex külön kezelve)
% http://tex.stackexchange.com/a/47579/71109
\usepackage{ifxetex}
\usepackage{ifluatex}
\newif\ifxetexorluatex % a new conditional starts as false
\ifnum 0\ifxetex 1\fi\ifluatex 1\fi>0
   \xetexorluatextrue
\fi

\ifxetexorluatex
  \usepackage{fontspec}
\else
  \usepackage[T1]{fontenc}
  \usepackage[utf8]{inputenc}
  \usepackage[lighttt]{lmodern}
\fi

% LaTeX és csomagok beállításai

%--------------------------------------------------------------------------------------
% Betűtípus
%--------------------------------------------------------------------------------------
% %ez az alap               % Latin Modern
\usepackage{dejavu}
%\usepackage{kpfonts}        % Palatino Linotype alternatíva
% \usepackage{libertine}    % Times New Roman alternatíva
% egyéb fontok: https://tug.org/FontCatalogue/

%--------------------------------------------------------------------------------------
% Változók
%--------------------------------------------------------------------------------------
\def\szerzok{\szerzoA}

\makeatletter
\ifthenelse{\equal{\szerzoB}{}}{}{\g@addto@macro\szerzok{, \szerzoB}}
\ifthenelse{\equal{\szerzoC}{}}{}{\g@addto@macro\szerzok{, \szerzoC}}
\ifthenelse{\equal{\szerzoD}{}}{}{\g@addto@macro\szerzok{, \szerzoD}}
\makeatother

\author{\szerzok}
\title{\cim}

%--------------------------------------------------------------------------------------
% Oldal layout
%--------------------------------------------------------------------------------------
% we need to redefine the pagestyle plain
% another possibility is to use the body of this command without \fancypagestyle
% and use \pagestyle{fancy} but in that case the special pages
% (like the ToC, the References, and the Chapter pages)remain in plane style

\pagestyle{plain}
\geometry{paper=a4paper}
\geometry{inner=30mm, outer=20mm, top=20mm, bottom=25mm}

\setcounter{tocdepth}{3}
\setcounter{secnumdepth}{3}

%--------------------------------------------------------------------------------------
% Szövegbeállítások
%--------------------------------------------------------------------------------------

\onehalfspacing                 % másfeles sorköz
%\singlespacing					% (egyszeres sorköz)
\selectlanguage{magyar}         % nyelv (magyar.ldf)
\setlength{\parindent}{2em}     % bekezdés behúzás mértéke (magyaros: 2em, angolos: 0)
\setlength{\parskip}{8pt}       % bekezdések közti térköz
\frenchspacing                  % mondatok közt is csak 1 "szóköznyi" kihagyás legyen

% Ha inkább angol beállítások kellenek:
	% \selectlanguage{english}
	% \setlength{\parindent}{0em}
	% \setlength{\parskip}{0.5em}
	% \nonfrenchspacing
	% \renewcommand{\figureautorefname}{Figure}
	% \renewcommand{\tableautorefname}{Table}
	% \renewcommand{\partautorefname}{Part}
	% \renewcommand{\chapterautorefname}{Chapter}
	% \renewcommand{\sectionautorefname}{Section}
	% \renewcommand{\subsectionautorefname}{Section}
	% \renewcommand{\subsubsectionautorefname}{Section}


\sloppy                                 % Margón túllógó sorok tiltása.
\widowpenalty=10000 \clubpenalty=10000  % A fattyú- és árvasorok elkerülése
\def\hyph{-\penalty0\hskip0pt\relax}    % Kötőjeles szavak elválasztásának engedélyezése


%--------------------------------------------------------------------------------------
% Bibliográfia stílusa
%--------------------------------------------------------------------------------------
\bibliographystyle{include/huplain}


%--------------------------------------------------------------------------------------
% Linkek és pdf címkék (hyperref)
%--------------------------------------------------------------------------------------
\hypersetup{
    % bookmarks=true,               % show bookmarks bar?
    unicode=true,                   % non-Latin characters in Acrobat's bookmarks
    pdftitle={\cim},                % title
    pdfauthor={\szerzok},           % author
    pdfsubject={},                  % subject of the document
    pdfcreator={},                  % creator of the document
    pdfproducer={},                 % producer of the document
    pdfkeywords={},                 % list of keywords (separate then by comma)
    pdfnewwindow=true,              % links in new window
    colorlinks=true,                % false: boxed links; true: colored links
    linkcolor=black,                % color of internal links
    citecolor=black,                % color of links to bibliography
    filecolor=black,                % color of file links
    urlcolor=black                  % color of external links
}


%--------------------------------------------------------------------------------------
% listings csomag (forráskód megjelenítés) beállításai
%--------------------------------------------------------------------------------------
\definecolor{lightgray}{rgb}{0.95,0.95,0.95}
\lstset{
	basicstyle=\scriptsize\ttfamily, % print whole listing small
	keywordstyle=\color{black}\bfseries, % bold black keywords
	identifierstyle=, % nothing happens
	% default behavior: comments in italic, to change use
	% commentstyle=\color{green}, % for e.g. green comments
	stringstyle=\scriptsize,
	showstringspaces=false, % no special string spaces
	aboveskip=3pt,
	belowskip=3pt,
	backgroundcolor=\color{lightgray},
	columns=flexible,
	keepspaces=true,
	escapeinside={(*@}{@*)},
	captionpos=b,
	breaklines=true,
	frame=single,
	float=!ht,
	tabsize=2,
	literate=*
		{á}{{\'a}}1	{é}{{\'e}}1	{í}{{\'i}}1	{ó}{{\'o}}1	{ö}{{\"o}}1	{ő}{{\H{o}}}1	{ú}{{\'u}}1	{ü}{{\"u}}1	{ű}{{\H{u}}}1
		{Á}{{\'A}}1	{É}{{\'E}}1	{Í}{{\'I}}1	{Ó}{{\'O}}1	{Ö}{{\"O}}1	{Ő}{{\H{O}}}1	{Ú}{{\'U}}1	{Ü}{{\"U}}1	{Ű}{{\H{U}}}1
}


%--------------------------------------------------------------------------------------
% Saját parancsok
%--------------------------------------------------------------------------------------
\newcommand{\code}[1]{{\upshape\ttfamily\scriptsize\indent #1}}
% A hivatkozások közt így könnyebb DOI-t megadni: \doi{xxxx.yyyy}
\newcommand{\doi}[1]{DOI: \href{http://dx.doi.org/\detokenize{#1}}{\raggedright{\texttt{\detokenize{#1}}}}}

\DeclareMathOperator*{\argmax}{arg\,max}
\DeclareMathOperator{\sign}{sgn}
\DeclareMathOperator{\rot}{rot}

%--------------------------------------------------------------------------------------
% Képaláírás stílusa
%--------------------------------------------------------------------------------------
\captionsetup[figure]{
	width=.75\textwidth,
	aboveskip=10pt}

\renewcommand{\captionlabelfont}{\it}
\renewcommand{\captionfont}{\footnotesize\it}

%--------------------------------------------------------------------------------------
% Elválasztás kivételei
%--------------------------------------------------------------------------------------
\hyphenation{Shakes-peare Mar-seilles ár-víz-tű-rő tü-kör-fú-ró-gép}

%--------------------------------------------------------------------------------------
% Dokumentum törzse
%--------------------------------------------------------------------------------------

\begin{document}
\pagenumbering{arabic}        % Oldalak számozása

% Címoldal
\hypersetup{pageanchor=false}

%--------------------------------------------------------------------------------------
%	Címoldal [Title page]
%--------------------------------------------------------------------------------------
\begin{titlepage}
\begin{center}
\includegraphics[width=60mm]{images/kozmosz-logo}\\
\vspace{0.3cm}

\textbf{BME vagy valami ilyesmi}

\vspace{0.1cm}

Egyéb szöveg amit ide akarsz írni

\vspace{5cm}

{\huge \MakeUppercase{\cim}}\\
\vspace{0.8cm}
{\LARGE \alcim}

\vspace{4cm}

{\large \textit{Készítette}}
\vspace{0.3cm}

{\large
    \szerzoA

    \szerzoB

    \szerzoC

    \szerzoD
}

\vfill
{\LARGE Budapest, \datum}
\end{center}
\end{titlepage}

% Tartalomjegyzék
\tableofcontents\vfill


\section{Az Apollo-program}

Az Apollo-program az Egyesült Államok második – a hosszas előkészítő fázisa miatt a repülések sorrendjét tekintve harmadik – emberek részvételével végrehajtott űrprogramja volt, amely 1961 és 1972 között zajlott. A program célja kettős volt: a fő célként az ember Holdra juttatása fogalmazódott meg, mögöttes politikai célként pedig a hidegháború által életre hívott űrversenyben az USA vesztes pozíciójának megfordítása, a nemzeti presztízs helyreállítása volt a célkitűzés \cite{Wettl2004}.

A holdprogram hivatalos bejelentése 1961. május 25-én John F. Kennedy elnök kongresszusi beszédében történt, ezt tekintjük az Apollo-program hivatalos kezdetének \cite{Cindy1986}. A beszédben Kennedy 9 éves határidőt tűzött ki a program megvalósításra. A célt 1969. július 21-én, az Apollo–11 űrhajósainak, Neil Armstrongnak és Buzz Aldrinnak a Holdra lépésével sikerült teljesíteni. Őket még további öt űrhajóspáros követte, így összesen hat sikeres holdra szállást teljesítettek a NASA űrhajósai. Armstrongék holdra szállását négy, űrhajósokkal végrehajtott tesztrepülés előzte meg, míg egy sikertelen holdutazás is része volt a programnak. A repülések mellett egy tragédia is beárnyékolta az Apollo-programot, az első tesztrepülés előtti előkészületek közben három űrhajós, Gus Grissom, Ed White és Roger Chaffee halt meg az űrkabinjukban kitört tűz következtében.

\begin{figure}[htb]
  \centering
  \includegraphics[width=0.5\textwidth]{images/foto}
  \caption{A Hold meghódítása}
  \label{fig:foto}
\end{figure}

A repüléseket egy speciális űrhajórendszerrel hajtották végre, amely az Apollo típusú űrhajóból és a holdra szállás kulcsának számító holdkompból állt, hordozóeszközként pedig szintén speciálisan a feladathoz tervezett Saturn V és Saturn IB rakétákat használtak. A program hardverét később sikerrel alkalmazták más űrkutatási programokban is, így a Skylab-programban és az Szojuz–Apollo repülésen is.

A program 1972-ben fejeződött be, azóta egyetlen embert szállító űrhajó sem hagyta el az alacsony Föld körüli pályát. Az űrhajósok által visszahozott kőzetminták és a kihelyezett műszerek mérései forradalmi változásokat hoztak a Naprendszer történetének, kialakulásának megismerésében, a Föld-Hold rendszer fejlődéstörténetének ismereteiben.

Az Egyesült Államok az Apollo-programra több mint 19,5 milliárd dollárt költött.

\section{Előzmények}

\subsection{Korai elképzelések}

Röviddel Konsztantyin Ciolkovszkij rakétaelvet megalkotó munkáinak publikálása után megjelentek az első elképzelések a világűr, azon belül pedig a legkézenfekvőbb cél, a Hold elérésére. Ezek közül a később legértékesebbnek bizonyult elképzelés Jurij Kondratyuk munkája volt, aki az anyaűrhajó/holdkomp rendszerű holdra szállás elméleti alapjait fektette le. Fantasztikus elképzelésekben később sem volt hiány, ilyen volt a három legkomplexebb ismeretekkel rendelkező tudós, Wernher von Braun rakétamérnök, Fred Whiplle és Willy Ley csillagászok publikációja a Holdutazás megvalósításáról a Colier's magazinban,[5] ami utópiának tűnt az 1950-es évek elején. Ebben az elképzelésben 15 űrhajós utazott volna, három óriási űrhajóval, amelyeket Föld körüli pályán szereltek volna össze, majd töltöttek volna fel üzemanyaggal. A háromból az egyik teherűrhajó lett volna, aminek célja pusztán a visszaútra szükséges üzemanyag szállítása lett volna. A leszállás után 6 hétig lettek volna az űrhajósok a Hold felszínén, napfelkeltétől kezdve egészen a következő – négy földi hétig tartó – nappal végéig. A kiürült üzemanyagtartályok hengereit lakómodulokká lehetett volna alakítani, egyfajta Hold-bázist létrehozva. Az építkezést daruk, és egy tíztonnás traktor (amit a későbbi felfedező utak járműveként is lehetett volna használni) könnyítette volna meg. Ennek az elképzelésnek a megvalósításához több száz tonna anyagot kellett volna megmozgatni, mindezt abban az időben, amikor az 1 tonnás Mercury-űrkabint sem tudták biztonságosan Föld körüli pályára állítani.[6]

\begin{table}[htb]
  \caption{Repülések}
  \centering
  \begin{tabular}{ccc}
    \toprule
    Repülés & Dátum & Időtartam \\
    \midrule
    Apollo-7 & 1968 október 11. & 11 nap \\
    Apollo-8 & 1968 november 21. & 7 nap \\
    Apollo-9 & 1969 március 3 &  10 nap \\
    \bottomrule
  \end{tabular}
  \label{tab:repulesek}
\end{table}

\subsection{Közvetlen előzmények}

\subsubsection{„Szputnyik-krízis”}

A hidegháború két egymással vetekedő nagyhatalma az 1950-es évek végén, a nemzetközi geofizikai év tudományos kísérlet- és rendezvénysorozatában találta meg azt az új területet, ahol kiterjeszthetik a technológiai versengésüket: az Egyesült Államok és a Szovjetunió egyaránt a világűrbe kívánta juttatni a maga űreszközét. A cél a másik fél fölötti technológiai fennhatóság bizonyítása volt. A versenyt a szovjetek nyerték, amikor 1957. október 4-én sikerrel juttatták az űrbe az addig titokban fejlesztett rakétájukkal a Szputnyik–1-et. Az USA-ban ezt szinte háborús hadüzenetként értelmezték (a szovjetek érdemi üzenete a műhold Föld körüli pályára állításával az volt: ha körbe tudunk juttatni a Földön egy tárgyat, akkor a Föld bármely pontját elérhetjük, bármely pontját képesek vagyunk bombázni).

Az USA vezetése elfogadta a szovjet kihívást és koncentrált erőfeszítést tett az űrteljesítmények területén az ellenfél előnyének behozására. E cél elérésére létrehozták a NASA-t, amely az összes korábban létrehozott repülési és űrhajózási kísérleti műhelyt vonta egy szervezetbe, hogy az erőket egyesítve minél hatékonyabban érjék el a kitűzött célt. 


\section{Tippek és trükkök}

\subsubsection*{Kiemelés}
A kiemelésre az ízléses megoldás az \verb|\emph{}| használata. Ez normál szövegben \emph{dőlt szövegként} jelenik meg. Az \underline{aláhúzás} és a \textbf{félkövér szöveg} használata kerülendő. Lehet még ,,idézőjeleket'' használni, ezt kicsit furán kell írni: két vessző, és két aposztrof.

Kódhoz használható a \texttt{monospace font} a \verb|\texttt{}| paranccsal, vagy a \verb!\verb||! parancs. A \verb!\verb||! határoló ,,zárójelei'' bármilyen karakterből állhatnak, ami nincs a verb belsejében. Pl. \verb|\verb!!|, \verb|\verb``|.

\subsubsection*{Példa felsorolásra}
\begin{itemize}
  \item Neil A. Armstrong
  \item Michael Collins
  \item Edwin E. Aldrin Jr.
\end{itemize}

\subsubsection*{Én így szoktam hosszabb szöveget felsorolásba rakni}
\begin{itemize}
  \item \textbf{Vivamus nec lectus} Ut tellus rhoncus blandit. Vivamus suscipit lobortis turpis, a eleifend sem pellentesque eget. Sed id posuere urna. Ut nec orci sapien. Aenean pharetra, nulla vitae mattis fermentum, arcu arcu scelerisque nisi, ut tristique velit elit et magna.
  \item \textbf{Cras consectetur} Purus sed ipsum pulvinar, eu bibendum nisl porta. Curabitur vehicula nisi rutrum, pretium lacus vel, sagittis tortor. Vivamus non sem at mauris consectetur ultricies ut non ligula. Ut justo mauris, lacinia id tempor eget, viverra vel orci.
  \item \textbf{Etiam luctus feugiat} Aliquam tempus faucibus eros facilisis faucibus. Pellentesque eget posuere metus. Cras in sapien est. Nunc ut mi at nisl commodo congue. Nam sodales non magna ac mattis. In malesuada euismod nisi. Fusce egestas varius orci, non posuere nisl. Sed porttitor erat quis dolor sodales placerat. In mattis ex dolor, fringilla cursus ligula consequat eu. Etiam eget neque ac est placerat elementum. In quis ipsum elit. 
\end{itemize}

\subsubsection*{Számozott lista}
\begin{enumerate}
  \item Neil A. Armstrong
  \item Michael Collins
  \item Edwin E. Aldrin Jr.
\end{enumerate} 

\subsubsection*{Irodalmi hivatkozás}
A hivatkozásokat BibTeX formátumban kell gyűjteni \cite{Lee1987}. Akadémiai oldalakról általában közvetlenül ilyet le lehet tölteni, de kézzel is könnyű elkészíteni \cite{Nasa}. A hivatkozások részleteit a \verb|bibliografia.bib| fájlba kell elhelyezni. Az ott megadott névvel hivatkozhatunk rá a szövegben a \verb|\cite| paranccsal \cite{Jeney2014}.

\subsubsection*{Kereszthivatkozás}
Az ábráknak, táblázatoknak, fejezeteknek megadhatunk egy \verb|\label| címkét. Ezzel tudunk rá hivatkozni a dokumentum más részeiben a \verb|\ref| paranccsal. Lásd \ref{fig:foto}. ábra, \ref{tab:repulesek}. táblázat. A labeleket érdemes típus szerint elnevezni, például \verb|fig:|, \verb|tab:|, \verb|sec:| előtaggal ellátni. 

Jó trükk még a hivatkozás előtti névelő (\emph{a} vagy \emph{az}) automatizálása. Ehhez a \verb|\ref| helyett az \verb|\aref| (kisbetűs) vagy \verb|\Aref| (nagybetűs) parancsokat kell használni. Részletek \aref{tab:repulesek}. táblázatban.


\subsubsection*{Példa forráskódra}
\begin{lstlisting}[language=C]
  #include <stdio.h>

  void main() {
      printf("Hello, Moon!");
  }
\end{lstlisting}

%--------------------------------------------------------------------------------------
% Bibliográfia
%--------------------------------------------------------------------------------------
\addcontentsline{toc}{section}{Irodalomjegyzék}
{\footnotesize \bibliography{bibliografia}}


%--------------------------------------------------------------------------------------
% Mellékletek
%--------------------------------------------------------------------------------------
\appendix
\addcontentsline{toc}{section}{Mellékletek}

\section{Kapcsolási rajzok}

\section{Röppálya-kalkuláció}



\end{document}
