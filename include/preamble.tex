
%--------------------------------------------------------------------------------------
% Betűtípus
%--------------------------------------------------------------------------------------
% %ez az alap               % Latin Modern
\usepackage{dejavu}
%\usepackage{kpfonts}        % Palatino Linotype alternatíva
% \usepackage{libertine}    % Times New Roman alternatíva
% egyéb fontok: https://tug.org/FontCatalogue/

%--------------------------------------------------------------------------------------
% Változók
%--------------------------------------------------------------------------------------
\def\szerzok{\szerzoA}

\makeatletter
\ifthenelse{\equal{\szerzoB}{}}{}{\g@addto@macro\szerzok{, \szerzoB}}
\ifthenelse{\equal{\szerzoC}{}}{}{\g@addto@macro\szerzok{, \szerzoC}}
\ifthenelse{\equal{\szerzoD}{}}{}{\g@addto@macro\szerzok{, \szerzoD}}
\makeatother

\author{\szerzok}
\title{\cim}

%--------------------------------------------------------------------------------------
% Oldal layout
%--------------------------------------------------------------------------------------
% we need to redefine the pagestyle plain
% another possibility is to use the body of this command without \fancypagestyle
% and use \pagestyle{fancy} but in that case the special pages
% (like the ToC, the References, and the Chapter pages)remain in plane style

\pagestyle{plain}
\geometry{paper=a4paper}
\geometry{inner=30mm, outer=20mm, top=20mm, bottom=25mm}

\setcounter{tocdepth}{3}
\setcounter{secnumdepth}{3}

%--------------------------------------------------------------------------------------
% Szövegbeállítások
%--------------------------------------------------------------------------------------

\onehalfspacing                 % másfeles sorköz
%\singlespacing					% (egyszeres sorköz)
\selectlanguage{magyar}         % nyelv (magyar.ldf)
\setlength{\parindent}{2em}     % bekezdés behúzás mértéke (magyaros: 2em, angolos: 0)
\setlength{\parskip}{8pt}       % bekezdések közti térköz
\frenchspacing                  % mondatok közt is csak 1 "szóköznyi" kihagyás legyen

% Ha inkább angol beállítások kellenek:
	% \selectlanguage{english}
	% \setlength{\parindent}{0em}
	% \setlength{\parskip}{0.5em}
	% \nonfrenchspacing
	% \renewcommand{\figureautorefname}{Figure}
	% \renewcommand{\tableautorefname}{Table}
	% \renewcommand{\partautorefname}{Part}
	% \renewcommand{\chapterautorefname}{Chapter}
	% \renewcommand{\sectionautorefname}{Section}
	% \renewcommand{\subsectionautorefname}{Section}
	% \renewcommand{\subsubsectionautorefname}{Section}


\sloppy                                 % Margón túllógó sorok tiltása.
\widowpenalty=10000 \clubpenalty=10000  % A fattyú- és árvasorok elkerülése
\def\hyph{-\penalty0\hskip0pt\relax}    % Kötőjeles szavak elválasztásának engedélyezése


%--------------------------------------------------------------------------------------
% Bibliográfia stílusa
%--------------------------------------------------------------------------------------
\bibliographystyle{include/huplain}


%--------------------------------------------------------------------------------------
% Linkek és pdf címkék (hyperref)
%--------------------------------------------------------------------------------------
\hypersetup{
    % bookmarks=true,               % show bookmarks bar?
    unicode=true,                   % non-Latin characters in Acrobat's bookmarks
    pdftitle={\cim},                % title
    pdfauthor={\szerzok},           % author
    pdfsubject={},                  % subject of the document
    pdfcreator={},                  % creator of the document
    pdfproducer={},                 % producer of the document
    pdfkeywords={},                 % list of keywords (separate then by comma)
    pdfnewwindow=true,              % links in new window
    colorlinks=true,                % false: boxed links; true: colored links
    linkcolor=black,                % color of internal links
    citecolor=black,                % color of links to bibliography
    filecolor=black,                % color of file links
    urlcolor=black                  % color of external links
}


%--------------------------------------------------------------------------------------
% listings csomag (forráskód megjelenítés) beállításai
%--------------------------------------------------------------------------------------
\definecolor{lightgray}{rgb}{0.95,0.95,0.95}
\lstset{
	basicstyle=\scriptsize\ttfamily, % print whole listing small
	keywordstyle=\color{black}\bfseries, % bold black keywords
	identifierstyle=, % nothing happens
	% default behavior: comments in italic, to change use
	% commentstyle=\color{green}, % for e.g. green comments
	stringstyle=\scriptsize,
	showstringspaces=false, % no special string spaces
	aboveskip=3pt,
	belowskip=3pt,
	backgroundcolor=\color{lightgray},
	columns=flexible,
	keepspaces=true,
	escapeinside={(*@}{@*)},
	captionpos=b,
	breaklines=true,
	frame=single,
	float=!ht,
	tabsize=2,
	literate=*
		{á}{{\'a}}1	{é}{{\'e}}1	{í}{{\'i}}1	{ó}{{\'o}}1	{ö}{{\"o}}1	{ő}{{\H{o}}}1	{ú}{{\'u}}1	{ü}{{\"u}}1	{ű}{{\H{u}}}1
		{Á}{{\'A}}1	{É}{{\'E}}1	{Í}{{\'I}}1	{Ó}{{\'O}}1	{Ö}{{\"O}}1	{Ő}{{\H{O}}}1	{Ú}{{\'U}}1	{Ü}{{\"U}}1	{Ű}{{\H{U}}}1
}


%--------------------------------------------------------------------------------------
% Saját parancsok
%--------------------------------------------------------------------------------------
\newcommand{\code}[1]{{\upshape\ttfamily\scriptsize\indent #1}}
% A hivatkozások közt így könnyebb DOI-t megadni: \doi{xxxx.yyyy}
\newcommand{\doi}[1]{DOI: \href{http://dx.doi.org/\detokenize{#1}}{\raggedright{\texttt{\detokenize{#1}}}}}

\DeclareMathOperator*{\argmax}{arg\,max}
\DeclareMathOperator{\sign}{sgn}
\DeclareMathOperator{\rot}{rot}

%--------------------------------------------------------------------------------------
% Képaláírás stílusa
%--------------------------------------------------------------------------------------
\captionsetup[figure]{
	width=.75\textwidth,
	aboveskip=10pt}

\renewcommand{\captionlabelfont}{\it}
\renewcommand{\captionfont}{\footnotesize\it}

%--------------------------------------------------------------------------------------
% Elválasztás kivételei
%--------------------------------------------------------------------------------------
\hyphenation{Shakes-peare Mar-seilles ár-víz-tű-rő tü-kör-fú-ró-gép}